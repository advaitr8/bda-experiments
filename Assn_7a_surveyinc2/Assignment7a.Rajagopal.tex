\documentclass{article}

\title{Assignment 7.a for \textbf{STATGR6103}\\
\large submitted to Professor Andrew Gelman}
\date{19 October 2016}
\author{Advait Rajagopal}

\usepackage{amsmath}
\usepackage{latexsym}
\usepackage{graphicx}
\usepackage{amsfonts}
\usepackage{wasysym}
\usepackage{amssymb}
\usepackage{mathrsfs}
\usepackage{multirow,array}
\usepackage{booktabs}
\usepackage{float}

\usepackage[a4paper,bindingoffset=0.2in,%
      left=1in,right=1in,top=1in,bottom=1in,%
          footskip=.25in]{geometry}
\DeclareMathAlphabet{\mathpzc}{OT1}{pzc}{m}{it}
\linespread{1.3}
\usepackage{listings}
\usepackage[most]{tcolorbox}
\usepackage{inconsolata}
\newtcblisting[auto counter]{sexylisting}[2][]{sharp corners, 
    fonttitle=\bfseries, colframe=gray, listing only, 
    listing options={basicstyle=\ttfamily,language=java}, 
    title=Listing \thetcbcounter: #2, #1}


\begin{document}
  \maketitle
\section{Question 1}
\textbf{Continued from Assignment 6.b}\\
Continuing the applied problem on survey incentives from the previous assignment, fit some hierarchical regression models with strong priors, so as to get stable estimates of the effects of different sorts of incentives under different conditions. What can you conclude about the effects of survey incentives, and how do your conclusions differ from what we said in our 2003 paper? 
\section*{Answer 1}
Picking up the analysis where it was left at the previous assignment I started by replicating Gelman et al.(2003)\footnote{http://www.stat.columbia.edu/~gelman/research/published/jbes01m045r3.pdf} conclusions to get a sound understanding of the data and explore it completely. As mentioned in the previous assignment, the initial analysis ignored the hierarchical structure of the data and failed to capture survey level variation. This was done much more convincingly in this assignment.\\ Analogous to Gelman et.al(2003), the model specifications are as given below;
$$y_i \sim N((X\beta)_i + \alpha_{j(i)}, \sigma^2 + V_i)$$
Where $y_i$ is the response rate in the population, $X$ is a matrix of predictors and $\beta$ represents a vector of regression coefficients and $\alpha_{j(i)}$ is a random effect at the survey level (for each of 39 surveys) and moreover we model the $\alpha$ using a normal distribution;
$$ \alpha_j \sim N(0, \tau^2)$$
We are able to use the normal approximation to the binomial model because the response rates are far enough from 0 and 1 in each case. We also incorporate the variance in the binomial distribution which is $V_i = y_i(1-y_i)/N_i$ and thus arrive at the linear approximation described above. The main predictors are incentive (present or absent), value of incentive (if present), mode of administering the survey (telephone or face to face), burden (high or low), timing of incentive(pre or post survey) and form of incentive (gift or cash).\\
The posterior estimates of all the parameters for each model are presented in Table 1. Model I is a simple linear regression model with no interaction. Models II and III have second and third order interaction terms respectively. Model IV has all individual terms, second and third order interactions. It is the most extensive model but as the question says is accused of being too noisy. So I use some tight priors to obtain \textbf{stable} and consistent estimates of the posterior parameters. A further explanation will be more useful once each model and its specifications are presented.

\begin{table}[H]
\caption {Posterior means(standard deviation) of regression coefficients}
\vspace{2mm}
\def\arraystretch{1.5}
\centering \begin{tabular}{c c c c c} 
\hline\hline 
\vspace{1mm}
& Model I & Model II & Model III  & Model IV \\ [0.5ex] 
\hline 
Intercept & 60 (3) & 63.4 (3.3) & 61.1 (3.4) & 63 (2.9)\\
Incentive & 5 (1)  & 4 (.4) & 2.1 (1.5)  & 5.95 (.15)\\ 
Mode & 15 (7) & 17 (5) & 15.4 (7.4) & 18.1 (.15) \\
Burden & -7 (6) & -8.9 (4) & -8.6 (4) & -9.9 (.15)\\
Mode $\times$ Burden &  & -5.7 (8)  & -9 (8) & -4.9 (.14) \\
Incentive $\times$ Value &  & 0.3 (0.1) & 0.3 (0.1) & 0.29 (0.08) \\
Incentive $\times$ Timing &  & 1.5 (1.2) & 2 (2) & -0.18 (.15)\\
Incentive $\times$ Form &  & -1 (1.1) & -2 (2) & -1.19 (.13) \\
Incentive $\times$ Mode &  & -2.3 (1.2) & -2 (2) & 7.8 (.14) \\
Incentive $\times$ Burden  & & 4.9 (2)  & 4.7 (2) & -5.3 (.15)\\
Incentive $\times$ Value $\times$ Timing &  &  & 0.4 (0.2) & 0.48 (.12)\\ 
Incentive $\times$ Value $\times$ Burden  &  &  & -0.2 (0.1)  & 1.17 (.1)  \\
Incentive $\times$ Timing $\times$ Burden  &  &  & & 10.99 (0.15) \\
Incentive $\times$ Value $\times$ Form  &  &  & & 0.1 (0.12) \\
Incentive $\times$ Value $\times$ Mode  &  &  & & -1.26 (0.1)\\
Incentive $\times$ Timing $\times$ Form  &  &  & & 9.87 (.15)\\
Incentive $\times$Timing $\times$ Mode  &  &  & & -17.4 (.15) \\
Incentive $\times$ Form $\times$ Mode  &  &  & & -0.31 (.15) \\
Incentive $\times$ Form $\times$ Burden  &  &  & & -5.9 (.15)\\
Incentive $\times$ Mode $\times$ Burden  &  &  & & -5.81 (.15) \\
\hline 
$\sigma$  & 4 (1) & 3.4 (0.4) & 3.3 (0.5) & 4.03 (0.5)\\
$\tau$ & 18 (2) & 19 (2.3) & 18.3 (2.3) & 18.4 (2.3)\\
\end{tabular}
\end{table}

\newpage
\begin{table}[H]
\caption {Sign convention for variables}
\vspace{2mm}
\def\arraystretch{1.5}
\centering \begin{tabular}{c c c } 
\hline\hline 
\vspace{1mm}
& Variable & Sign \\ [0.5ex] 
\hline 
Mode & Face/Telephone & +/- \\
Burden & High/Low& +/- \\ 
Timing & Before/After& +/-  \\
Form & Cash/Gift & +/-\\
\hline 
\end{tabular}
\end{table}
\subsection{Model I and Model II}
Model I simply has a few of the predictors namely, a dummy for incentive and indicators for whether the incentive was given before or after the survey as well as the burden level of the survey. As expected, the burden has a negative coefficient on the response rate. Cash for example has a positive effect and so on (refer Table 2 for sign convention). However as it was stated in Assignment 6.b, our variables of interest are the ``dollar value of incentive" and the resulting response rate. So we are mostly interested in fixing interaction at 1 and observing the interaction of incentive with all other variables and in particular ``dollar value". This is to say we are mainly concerned with incentivized cases. Model 2 has second-order interactions as described in Table 1.
Thus ultimately it is only the intercept that varies for each of the 16 cases (changing mode, burden, time and form) amongst all possible signs and the resultant relationship between value and incentive is plotted in Figure 2.
 \begin{figure}[H]
\centering
\includegraphics[width = 14cm, height = 10cm]{model2.png}
\caption{$f$ is face to face and $t$ is telephone, $h$ is high and $l$ is low burden, $b$ is before and $a$ is after survey, $c$ is cash and $g$ is gift.}
\label{deltat}
\end{figure}
\subsection{Model III}
Model III has some second order and third order interaction terms. However if we run the model without any priors on the regression coefficients $\beta$ we observe some very noisy estimates. A little noise is a lot since these are response rates measured in percentages (or proportions) and small changes are meaningful. For instance the coefficient for the interaction between Mode and Burden has a standard deviation of 8 for a value that is -9. This is too extreme to be meaningful. So Model III is good for analyzing interactions but it can definitely be done better by including all possible interactions (see Model IV) and using some priors. However in this case I estimate the regression coefficients with uniform priors and plot the related results below. In Figure 2 I summarize Model III, and Figure 2 is similar to the Figure 2 in Gelman et. al(2003).
 \begin{figure}[H]
\centering
\includegraphics[width = 14cm, height = 10cm]{model3.png}
\caption{Red and blue lines indicate time of incentive (before and after respectively). Solid lines are for cash incentives and dotted lines are for gifts.}
\label{deltat}
\end{figure}
It is clear as soon as we examine Figure 2 that the red lines slope upward much faster than the blue lines meaning that pre-paying is always better than paying after the survey for response rates. Also telephone surveys seem to have higher response rates than face to face surveys conditional on incentive.
\subsection{Model IV}
There are several difficulties with the meta-analysis performed so far. Data is sparse since we have only 101 experimental conditions to estimate potentially six levels of interaction which further combine to form 35 linear predictors. Factors other than incentive are not assigned in a random manner. There is clearly a hierarchical structure in the data and this is handled in all models with survey level random variation. However through the process of fitting more and more interaction terms it becomes clear that the estimates are not stable and often the sign associated with an interaction predictor reverses or is estimated with a huge standard error. This inconsistency leads us to abandon a more elaborate model with greater explanatory power for a simpler one which is reproducible consistently. So I have used some tight priors. I use the uniform mean estimates from the previous models and Model IV run without priors as the mean but I use a standard deviation of 1.5 which I believe to be sufficiently large or large enough to not dominate the data but narrow enough to strongly guide the chains. It is interesting that the model converges with R-hat close to 1 and the ultimate standard deviation associated with each of the regression coefficients becomes much smaller than that of Gelman et.al(2003)\footnote{See Table 1 for results of the posterior estimates of the regression coefficients.}. It is advantageous that we are now able to include all meaning full three way interactions with `incentive' and this helps in the ultimate goal of modeling the relationship between ``value'' and response rate. On the whole my conclusion matches Gelman et.al(2003) about the positive relationship between the two but Model IV is significantly improved with less noisy estimates. Pre-paid incentives are better than post paid incentives and cash is more effective than gifts. Also we observe some interesting sign changes, for example, the interaction between incentive and mode now yields a positive coefficient suggesting that face to face cash payments would increase the response rate.

\section{Code}
\subsection{R Code}
\begin{sexylisting}{R Code}
# #Read in data
library(rstan)
rstan_options(auto_write = TRUE)
options(mc.cores = parallel::detectCores())
rm(list = ls())
setwd("/Users/Advait/Desktop/New School/Fall16/BDA/Class12")
inc <- read.table("http://www.stat.columbia.edu/~
gelman/bda.course/incentives_data_clean.txt",
                        skip = 12)
str(inc)
inc$id <- c(as.factor(inc$sid))
#Model1 - AG
incentive <- inc$I
mod <- inc$m
burden <- inc$b
y <- inc$r
id <- inc$id
V <- NULL
for (i in 1:101){
  V[i] <- (inc$r[i]*(1-inc$r[i]))/inc$basen[i]
}
V
##fit model 1
stanc("7a.stan")
fit1 <- stan("7a.stan", data = list("y", "incentive", "mod",
                                    "burden","V","id" ), 
             iter = 5000, chains = 3)
print(fit1)
##Model2 - AG
str(inc)
v <- inc$v
time <- inc$t
form <- inc$f
stanc("7a_model2.stan")
fit2 <- stan("7a_model2.stan", data = list("y","incentive","mod",
"burden","v","time","form","V","id"), iter = 1000, chains=3)
print(fit2, digits_summary = 3)
\end{sexylisting}

\begin{sexylisting}{R Code contd.}
##Model3 - AG
stanc("7a_model3.stan")
fit3 <- stan("7a_model3.stan", data = list("y","incentive","mod",
"burden","v","time","form","V","id"), iter = 1000, chains=3)
print(fit3, digits_summary = 3)
ext.3 <- extract(fit3)
##Model4 - AG
stanc("7a_model4.stan")
fit4 <- stan("7a_model4.stan", data = list("y","incentive","mod",
"burden","v","time","form","V","id"), iter = 5000, chains=3)
print(fit4, digits_summary = 4)
####
#Plot Model 2
par(mfcol = c(1,1))
money <- c(0:120)
plot(money, 0.6995 + 0.0015*money, lty =3, col = "green", type = "l",
     ylim = c(0.4,0.9), 
     main = "Incentive interacting with other variables",
     cex.main = 0.8, 
     ylab = "Response rate", xlab = "Dollar incentive",
     cex.lab = 0.7, lwd = 2)
points(inc$v[inc$I==1], inc$r[inc$I==1], 
pch = 16, cex = .4, col = "black")
lines(0.6895 + 0.0015*money, 
lty =3, col = "blue", lwd = 2)#FHBG
lines(0.6465 + 0.0015*money, 
lty =3, col = "brown", lwd = 2)#FHAG
lines(0.5975 + 0.0015*money, 
lty =3, col = "darkgray", lwd = 2)#FLAG
lines(0.6405 + 0.0015*money, 
lty =3, col = "yellow", lwd = 2)#FLBG
lines(0.6075 + 0.0015*money, 
lty =3, col = "maroon", lwd = 2)#FLAC
lines(0.6695 + 0.0015*money, 
lty =3, col = "darkgreen", lwd = 2)#THAG
lines(0.6635 + 0.0015*money, 
lty =3, col = "darkblue", lwd = 2)#TLBG
lines(0.6505 + 0.0015*money, 
lty =3, col = "gray", lwd = 2)#FLBC
lines(0.6565 + 0.0015*money, 
lty =3, col = "purple", lwd = 2)#FHAC
\end{sexylisting}

\begin{sexylisting}{R Code contd.}
lines(0.7125 + 0.0015*money, 
lty =3, col = "aquamarine3", lwd = 2)#THBG
lines(0.6795 + 0.0015*money, 
lty =3, col = "blueviolet", lwd = 2)#THAC

lines(0.7225 + 0.0015*money, 
lty =3, col = "brown1", lwd = 2)#THBC
lines(0.6735 + 0.0015*money, 
lty =3, col = "darkgoldenrod", lwd = 2)#TLBC
lines(0.6305 + 0.0015*money, 
lty =3, col = "cornflowerblue", lwd = 2)#TLAC
lines(0.6205 + 0.0015*money, 
lty =3, col = "chocolate", lwd = 2)#TLAG
legend(100,0.73,legend = c("fhbc", "fhbg","fhag","flag",
                           "flbg", "flac","thag", "tlbg",
                           "flbc", "fhac", "thbg","thac",
                           "thbc", "tlbc", "tlac", "tlag"),
        lty = 3, col = c("green", "blue", "brown", "darkgray",
                        "yellow", "maroon", "darkgreen", "darkblue",
                        "gray", "purple", "aquamarine3", "blueviolet",
                        "brown1", "darkgoldenrod", 
                        "cornflowerblue","chocolate"), 
                        bty = "n", cex = 1)
\end{sexylisting}
\begin{sexylisting}{R Code contd.}                                                
# ##################
# Plots model 3

x <- seq(0, 120, 1)

par(mfrow = c(2, 2), mar = c(2, 2, 2, 2))

#Plot Low Burden telephone surveys
plot(x, mean(ext.3$b0) + 
       mean(ext.3$b1)*1 + 
       mean(ext.3$b3)*(-0.5) +
       mean(ext.3$b2)*(-0.5) +
       mean(ext.3$b4)*(-0.5)*(-0.5) +
       mean(ext.3$b5)*x +
       mean(ext.3$b6)*(0.5) +
       mean(ext.3$b7)*0.5 +
       mean(ext.3$b8)*(-0.5) +
       mean(ext.3$b9)*(-0.5) +
       mean(ext.3$b10)*x*(0.5) +
       mean(ext.3$b11)*x*(-0.5) -
       (mean(ext.3$b0) + 
          mean(ext.3$b1)*0 + 
          mean(ext.3$b3)*(-0.5) +
          mean(ext.3$b2)*(-0.5) +
          mean(ext.3$b4)*(-0.5)*(-0.5)),
     ylab = "Diff in Response Rate", xlab = "Incentive",
     type = "l", col = "red",
     ylim = c(-0.1, 0.5), 
     main = "Low Burden Phone",
     cex.main = 0.8)
abline(h = 0, col = "grey", lty = 2)

lines(x, mean(ext.3$b0) + 
        mean(ext.3$b1)*1 + 
        mean(ext.3$b3)*(-0.5) +
        mean(ext.3$b2)*(-0.5) +
        mean(ext.3$b4)*(-0.5)*(-0.5) +
        mean(ext.3$b5)*x +
        mean(ext.3$b6)*(0.5) +
        mean(ext.3$b7)*(-0.5) +
        mean(ext.3$b8)*(-0.5) +
        mean(ext.3$b9)*(-0.5) +
        mean(ext.3$b10)*x*(0.5) +
        mean(ext.3$b11)*x*(-0.5) -
\end{sexylisting}
\begin{sexylisting}{R Code contd}
        (mean(ext.3$b0) + 
           mean(ext.3$b1)*0 + 
           mean(ext.3$b3)*(-0.5) +
           mean(ext.3$b2)*(-0.5) +
           mean(ext.3$b4)*(-0.5)*(-0.5)),
      ylab = "Response Rate", xlab = "Incentive",
      type = "l", col = "red", lty = 2, 
      ylim = c(-0.1, 1))

lines(x, mean(ext.3$b0) + 
        mean(ext.3$b1)*1 + 
        mean(ext.3$b3)*(-0.5) +
        mean(ext.3$b2)*(-0.5) +
        mean(ext.3$b4)*(-0.5)*(-0.5) +
        mean(ext.3$b5)*x +
        mean(ext.3$b6)*(-0.5) +
        mean(ext.3$b7)*(0.5) +
        mean(ext.3$b8)*(-0.5) +
        mean(ext.3$b9)*(-0.5) +
        mean(ext.3$b10)*x*(-0.5) +
        mean(ext.3$b11)*x*(-0.5) -
        (mean(ext.3$b0) + 
           mean(ext.3$b1)*0 + 
           mean(ext.3$b3)*(-0.5) +
           mean(ext.3$b2)*(-0.5) +
           mean(ext.3$b4)*(-0.5)*(-0.5)),
      ylab = "Response Rate", xlab = "Incentive",
      type = "l", col = "blue", 
      ylim = c(0, 1))

lines(x, mean(ext.3$b0) + 
        mean(ext.3$b1)*1 + 
        mean(ext.3$b3)*(-0.5) +
        mean(ext.3$b2)*(-0.5) +
        mean(ext.3$b4)*(-0.5)*(-0.5) +
        mean(ext.3$b5)*x +
        mean(ext.3$b6)*(-0.5) +
        mean(ext.3$b7)*(-0.5) +
        mean(ext.3$b8)*(-0.5) +
        mean(ext.3$b9)*(-0.5) +
        mean(ext.3$b10)*x*(-0.5) +
        mean(ext.3$b11)*x*(-0.5) -
\end{sexylisting}
 \begin{sexylisting}{R Code contd.}
        (mean(ext.3$b0) + 
           mean(ext.3$b1)*0 + 
           mean(ext.3$b3)*(-0.5) +
           mean(ext.3$b2)*(-0.5) +
           mean(ext.3$b4)*(-0.5)*(-0.5)),
      ylab = "Response Rate", xlab = "Incentive",
      type = "l", col = "blue", lty = 2, 
      ylim = c(0, 1))

#Plot Low Burden face to face surveys
plot(x, mean(ext.3$b0) + 
       mean(ext.3$b1)*1 + 
       mean(ext.3$b3)*(-0.5) +
       mean(ext.3$b2)*(0.5) +
       mean(ext.3$b4)*(0.5)*(-0.5) +
       mean(ext.3$b5)*x +
       mean(ext.3$b6)*(0.5) +
       mean(ext.3$b7)*0.5 +
       mean(ext.3$b8)*(0.5) +
       mean(ext.3$b9)*(-0.5) +
       mean(ext.3$b10)*x*(0.5) +
       mean(ext.3$b11)*x*(-0.5)-
       (mean(ext.3$b0) + 
          mean(ext.3$b1)*0 + 
          mean(ext.3$b3)*(-0.5) +
          mean(ext.3$b2)*(0.5) +
          mean(ext.3$b4)*(0.5)*(-0.5)),
     ylab = "Diff in Response Rate", xlab = "Incentive",
     type = "l", col = "red",
     ylim = c(-0.1, 0.5),
     main = "Low Burden Face-to-Face",
     cex.main = 0.8)
abline(h = 0, col = "grey", lty = 2)

lines(x, mean(ext.3$b0) + 
        mean(ext.3$b1)*1 + 
        mean(ext.3$b3)*(-0.5) +
        mean(ext.3$b2)*(0.5) +
        mean(ext.3$b4)*(0.5)*(-0.5) +
        mean(ext.3$b5)*x +
        mean(ext.3$b6)*(0.5) +
        mean(ext.3$b7)*(-0.5) +
        mean(ext.3$b8)*(0.5) +
        mean(ext.3$b9)*(-0.5) +
 \end{sexylisting}
 \begin{sexylisting}{R Code contd.}
        mean(ext.3$b10)*x*(0.5) +
        mean(ext.3$b11)*x*(-0.5) -
        (mean(ext.3$b0) + 
           mean(ext.3$b1)*0 + 
           mean(ext.3$b3)*(-0.5) +
           mean(ext.3$b2)*(0.5) +
           mean(ext.3$b4)*(0.5)*(-0.5)),
      ylab = "Response Rate", xlab = "Incentive",
      type = "l", col = "red", lty = 2, 
      ylim = c(-0.1, 1))

lines(x, mean(ext.3$b0) + 
        mean(ext.3$b1)*1 + 
        mean(ext.3$b3)*(-0.5) +
        mean(ext.3$b2)*(0.5) +
        mean(ext.3$b4)*(0.5)*(-0.5) +
        mean(ext.3$b5)*x +
        mean(ext.3$b6)*(-0.5) +
        mean(ext.3$b7)*(0.5) +
        mean(ext.3$b8)*(0.5) +
        mean(ext.3$b9)*(-0.5) +
        mean(ext.3$b10)*x*(-0.5) +
        mean(ext.3$b11)*x*(-0.5) -
        (mean(ext.3$b0) + 
           mean(ext.3$b1)*0 + 
           mean(ext.3$b3)*(-0.5) +
           mean(ext.3$b2)*(0.5) +
           mean(ext.3$b4)*(0.5)*(-0.5)),
      ylab = "Response Rate", xlab = "Incentive",
      type = "l", col = "blue", 
      ylim = c(0, 1))

lines(x, mean(ext.3$b0) + 
        mean(ext.3$b1)*1 + 
        mean(ext.3$b3)*(-0.5) +
        mean(ext.3$b2)*(0.5) +
        mean(ext.3$b4)*(0.5)*(-0.5) +
        mean(ext.3$b5)*x +
        mean(ext.3$b6)*(-0.5) +
        mean(ext.3$b7)*(-0.5) +
        mean(ext.3$b8)*(0.5) +
        mean(ext.3$b9)*(-0.5) +
        \end{sexylisting}
        \begin{sexylisting}{R Code contd.}
        mean(ext.3$b10)*x*(-0.5) +
        mean(ext.3$b11)*x*(-0.5) -
        (mean(ext.3$b0) + 
           mean(ext.3$b1)*0 + 
           mean(ext.3$b3)*(-0.5) +
           mean(ext.3$b2)*(0.5) +
           mean(ext.3$b4)*(0.5)*(-0.5)),
      ylab = "Response Rate", xlab = "Incentive",
      type = "l", col = "blue", lty = 2, 
      ylim = c(0, 1))


#Plot High Burden telephone surveys
plot(x, mean(ext.3$b0) + 
       mean(ext.3$b1)*1 + 
       mean(ext.3$b3)*(0.5) +
       mean(ext.3$b2)*(0.5) +
       mean(ext.3$b4)*(0.5)*(0.5) +
       mean(ext.3$b5)*x +
       mean(ext.3$b6)*(0.5) +
       mean(ext.3$b7)*0.5 +
       mean(ext.3$b8)*(0.5) +
       mean(ext.3$b9)*(0.5) +
       mean(ext.3$b10)*x*(0.5) +
       mean(ext.3$b11)*x*(0.5) -
       (mean(ext.3$b0) + 
          mean(ext.3$b1)*0 + 
          mean(ext.3$b3)*(0.5) +
          mean(ext.3$b2)*(0.5) +
          mean(ext.3$b4)*(0.5)*(0.5)),
     ylab = "Diff in Response Rate", xlab = "Incentive",
     type = "l", col = "red",
     ylim = c(-0.1, 0.5),
     main = "High Burden Phone",
     cex.main = 0.8)
abline(h = 0, col = "grey", lty = 2)

lines(x, mean(ext.3$b0) + 
        mean(ext.3$b1)*1 + 
        mean(ext.3$b3)*(0.5) +
        mean(ext.3$b2)*(0.5) +
        mean(ext.3$b4)*(0.5)*(0.5) +
        mean(ext.3$b5)*x +
\end{sexylisting}
\begin{sexylisting}{R Code contd.}
        mean(ext.3$b6)*(0.5) +
        mean(ext.3$b7)*(-0.5) +
        mean(ext.3$b8)*(0.5) +
        mean(ext.3$b9)*(0.5) +
        mean(ext.3$b10)*x*(0.5) +
        mean(ext.3$b11)*x*(0.5) -
        (mean(ext.3$b0) + 
           mean(ext.3$b1)*0 + 
           mean(ext.3$b3)*(0.5) +
           mean(ext.3$b2)*(0.5) +
           mean(ext.3$b4)*(0.5)*(0.5)),
      ylab = "Response Rate", xlab = "Incentive",
      type = "l", col = "red", lty = 2, 
      ylim = c(-0.1, 1))

lines(x, mean(ext.3$b0) + 
        mean(ext.3$b1)*1 + 
        mean(ext.3$b3)*(0.5) +
        mean(ext.3$b2)*(0.5) +
        mean(ext.3$b4)*(0.5)*(0.5) +
        mean(ext.3$b5)*x +
        mean(ext.3$b6)*(-0.5) +
        mean(ext.3$b7)*(0.5) +
        mean(ext.3$b8)*(0.5) +
        mean(ext.3$b9)*(0.5) +
        mean(ext.3$b10)*x*(-0.5) +
        mean(ext.3$b11)*x*(0.5) -
        (mean(ext.3$b0) + 
           mean(ext.3$b1)*0 + 
           mean(ext.3$b3)*(0.5) +
           mean(ext.3$b2)*(0.5) +
           mean(ext.3$b4)*(0.5)*(0.5)),
      ylab = "Response Rate", xlab = "Incentive",
      type = "l", col = "blue", 
      ylim = c(0, 1))

lines(x, mean(ext.3$b0) + 
        mean(ext.3$b1)*1 + 
        mean(ext.3$b3)*(0.5) +
        mean(ext.3$b2)*(0.5) +
        mean(ext.3$b4)*(0.5)*(0.5) +
        mean(ext.3$b5)*x +
        mean(ext.3$b6)*(-0.5) +
\end{sexylisting}
\begin{sexylisting}{R Code contd.}
        mean(ext.3$b7)*(-0.5) +
        mean(ext.3$b8)*(0.5) +
        mean(ext.3$b9)*(0.5) +
        mean(ext.3$b10)*x*(-0.5) +
        mean(ext.3$b11)*x*(0.5) -
        (mean(ext.3$b0) + 
           mean(ext.3$b1)*0 + 
           mean(ext.3$b3)*(0.5) +
           mean(ext.3$b2)*(0.5) +
           mean(ext.3$b4)*(0.5)*(0.5)),
      ylab = "Response Rate", xlab = "Incentive",
      type = "l", col = "blue", lty = 2, 
      ylim = c(0, 1))

#Plot High Burden face to face surveys
plot(x, mean(ext.3$b0) + 
       mean(ext.3$b1)*1 + 
       mean(ext.3$b3)*(0.5) +
       mean(ext.3$b2)*(-0.5) +
       mean(ext.3$b4)*(-0.5)*(0.5) +
       mean(ext.3$b5)*x +
       mean(ext.3$b6)*(0.5) +
       mean(ext.3$b7)*0.5 +
       mean(ext.3$b8)*(-0.5) +
       mean(ext.3$b9)*(0.5) +
       mean(ext.3$b10)*x*(0.5) +
       mean(ext.3$b11)*x*(0.5)-
       (mean(ext.3$b0) + 
          mean(ext.3$b1)*0 + 
          mean(ext.3$b3)*(0.5) +
          mean(ext.3$b2)*(-0.5) +
          mean(ext.3$b4)*(-0.5)*(0.5)),
     ylab = "Diff in Response Rate", xlab = "Incentive",
     type = "l", col = "red",
     ylim = c(-0.1, 0.5),
     main = "High Burden Face-to-Face",
     cex.main = 0.8)
abline(h = 0, col = "grey", lty = 2)

lines(x, mean(ext.3$b0) + 
        mean(ext.3$b1)*1 + 
        mean(ext.3$b3)*(0.5) +
        mean(ext.3$b2)*(-0.5) +
        \end{sexylisting}
        \begin{sexylisting}{R Code contd.}
        mean(ext.3$b4)*(-0.5)*(0.5) +
        mean(ext.3$b5)*x +
        mean(ext.3$b6)*(0.5) +
        mean(ext.3$b7)*(-0.5) +
        mean(ext.3$b8)*(-0.5) +
        mean(ext.3$b9)*(0.5) +
        mean(ext.3$b10)*x*(0.5) +
        mean(ext.3$b11)*x*(0.5) -
        (mean(ext.3$b0) + 
           mean(ext.3$b1)*0 + 
           mean(ext.3$b3)*(0.5) +
           mean(ext.3$b2)*(-0.5) +
           mean(ext.3$b4)*(-0.5)*(0.5)),
      ylab = "Response Rate", xlab = "Incentive",
      type = "l", col = "red", lty = 2, 
      ylim = c(-0.1, 1))

lines(x, mean(ext.3$b0) + 
        mean(ext.3$b1)*1 + 
        mean(ext.3$b3)*(0.5) +
        mean(ext.3$b2)*(-0.5) +
        mean(ext.3$b4)*(-0.5)*(0.5) +
        mean(ext.3$b5)*x +
        mean(ext.3$b6)*(-0.5) +
        mean(ext.3$b7)*(0.5) +
        mean(ext.3$b8)*(-0.5) +
        mean(ext.3$b9)*(0.5) +
        mean(ext.3$b10)*x*(-0.5) +
        mean(ext.3$b11)*x*(0.5) -
        (mean(ext.3$b0) + 
           mean(ext.3$b1)*0 + 
           mean(ext.3$b3)*(0.5) +
           mean(ext.3$b2)*(-0.5) +
           mean(ext.3$b4)*(-0.5)*(0.5)),
      ylab = "Response Rate", xlab = "Incentive",
      type = "l", col = "blue", 
      ylim = c(0, 1))

lines(x, mean(ext.3$b0) + 
        mean(ext.3$b1)*1 + 
        mean(ext.3$b3)*(0.5) +
        mean(ext.3$b2)*(-0.5) +
        \end{sexylisting}
        \begin{sexylisting}{R Code contd.}
        mean(ext.3$b4)*(-0.5)*(0.5) +
        mean(ext.3$b5)*x +
        mean(ext.3$b6)*(-0.5) +
        mean(ext.3$b7)*(-0.5) +
        mean(ext.3$b8)*(-0.5) +
        mean(ext.3$b9)*(0.5) +
        mean(ext.3$b10)*x*(-0.5) +
        mean(ext.3$b11)*x*(0.5) -
        (mean(ext.3$b0) + 
           mean(ext.3$b1)*0 + 
           mean(ext.3$b3)*(0.5) +
           mean(ext.3$b2)*(-0.5) +
           mean(ext.3$b4)*(-0.5)*(0.5)),
      ylab = "Response Rate", xlab = "Incentive",
      type = "l", col = "blue", lty = 2,
      ylim = c(0, 1))                      
\end{sexylisting}

\subsection{Stan Code}
\begin{sexylisting}{Stan Code Model 1}
data{
  real y[101];
  int incentive[101];
  real mod[101];
  real burden[101];
  real V[101];
  int id[101];
}
parameters{
  real b0;
  real b1;
  real b2;
  real b3;
  real <lower = 0> sigma;
  real alpha[39];
  real <lower = 0> tau;
}
model{
for (i in 1:101){
    y[i] ~ normal(b0 + b1*incentive[i] + b2*mod[i]
                  + b3*burden[i] + alpha[id[i]] , 
                  sqrt(pow(sigma,2) + V[i]));
  }
for (i in 1:39){
  alpha[i] ~ normal(0,tau);
}
}
\end{sexylisting}
\begin{sexylisting}{Stan Code Model 2}
data{
  real y[101];
  int incentive[101];
  real mod[101];
  real burden[101];
  real v[101];
  real time[101];
  real form[101];
  real V[101];
  int id[101];
}
parameters{
  real b0;
  real b1;
  real b2;
  real b3;
  real b4;
  real b5;
  real b6;
  real b7;
  real b8;
  real b9;
  real <lower = 0> sigma;
  real alpha[39];
  real <lower = 0> tau;
}
model{
  for (i in 1:101){
    y[i] ~ normal(b0 + b1*incentive[i] + b2*mod[i]
                  + b3*burden[i] + b4*(mod[i]*burden[i])  
                  + b5*(incentive[i]*v[i])
                  + b6*(incentive[i]*time[i])
                  + b7*(incentive[i]*form[i])
                  + b8*(incentive[i]*mod[i])
                  + b9*(incentive[i]*burden[i])
                  + alpha[id[i]] , 
                  sqrt(pow(sigma,2) + V[i]));
}
  for(i in 1:39)
alpha[i] ~ normal(0, tau);
}
\end{sexylisting}
\begin{sexylisting}{Stan Code Model 3}
data{
  real y[101];
  int incentive[101];
  real mod[101];
  real burden[101];
  real v[101];
  real time[101];
  real form[101];
  real V[101];
  int id[101];
}
parameters{
  real b0;
  real b1;
  real b2;
  real b3;
  real b4;
  real b5;
  real b6;
  real b7;
  real b8;
  real b9;
  real b10;
  real b11;
  real <lower = 0> sigma;
  real alpha[39];
  real <lower = 0> tau;
}
model{
for (i in 1:101){
    y[i] ~ normal(b0 + b1*incentive[i] + b2*mod[i]
                  + b3*burden[i] + b4*(mod[i]*burden[i])  
                  + b5*(incentive[i]*v[i])
                  + b6*(incentive[i]*time[i])
                  + b7*(incentive[i]*form[i])
                  + b8*(incentive[i]*mod[i])
                  + b9*(incentive[i]*burden[i])
                  + b10*(incentive[i]*v[i]*time[i])
                  + b11*(incentive[i]*v[i]*burden[i])
                  + alpha[id[i]] , sqrt(pow(sigma,2) + V[i]));
  }
for(i in 1:39){
    alpha[i] ~ normal(0, tau);}
}
\end{sexylisting}
\begin{sexylisting}{Stan Code Model 4}
data{
  real y[101];
  int incentive[101];
  real mod[101];
  real burden[101];
  real v[101];
  real time[101];
  real form[101];
  real V[101];
  int id[101];
}
parameters{
  real b0;
  real b1;
  real b2;
  real b3;
  real b4;
  real b5;
  real b6;
  real b7;
  real b8;
  real b9;
  real b10;
  real b11;
  real b12;
  real b13;
  real b14;
  real b15;
  real b16;
  real b17;
  real b18;
  real b19;
  real <lower = 0> sigma;
  real alpha[39];
  real <lower = 0> tau;
}
\end{sexylisting}
\begin{sexylisting}{Stan Code Model 4 contd.}
model{
for (i in 1:101){
    y[i] ~ normal(b0 + b1*incentive[i] + b2*mod[i]
                  + b3*burden[i] + b4*(mod[i]*burden[i])  
                  + b5*(incentive[i]*v[i])
                  + b6*(incentive[i]*time[i])
                  + b7*(incentive[i]*form[i])
                  + b8*(incentive[i]*mod[i])
                  + b9*(incentive[i]*burden[i])
                  + b10*(incentive[i]*v[i]*time[i])
                  + b11*(incentive[i]*v[i]*burden[i])
                  + b12*(incentive[i]*time[i]*burden[i])
                  + b13*(incentive[i]*v[i]*form[i])
                  + b14*(incentive[i]*v[i]*mod[i])
                  + b15*(incentive[i]*time[i]*form[i])
                  + b16*(incentive[i]*time[i]*mod[i])
                  + b17*(incentive[i]*form[i]*mod[i])
                  + b18*(incentive[i]*form[i]*burden[i])
                  + b19*(incentive[i]*mod[i]*burden[i])
                  + alpha[id[i]] , sqrt(pow(sigma,2) + V[i]));
b1 ~ normal(0.06,.015);
b2 ~ normal(0.018,.015);
b3 ~ normal(-.099,.015);
b4 ~ normal(-.049,.015);
b5 ~ normal(.0026,.015);
b6 ~ normal(-.002,.015);
b7 ~ normal(-.012,.015);
b8 ~ normal(.078,.015);
b9 ~ normal(-.052,.015);
b10 ~ normal(.0058,.015);
b11 ~ normal(.011,.015);
b12 ~ normal(.11,.015);
b13 ~ normal(.003,.015);
b14 ~ normal(-.012,.015);
b15 ~ normal(.099,.015);
b16 ~ normal(-.174,.015);
b17 ~ normal(-.003,.015);
b18 ~ normal(.059,.015);
b19 ~ normal(-.058,.015);
}
for(i in 1:39){
    alpha[i] ~ normal(0, tau);}
}
\end{sexylisting}





















\end{document}