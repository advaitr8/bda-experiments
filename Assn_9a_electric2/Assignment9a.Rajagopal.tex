\documentclass{article}

\title{Assignment 9.a for \textbf{STATGR6103}\\
\large submitted to Professor Andrew Gelman}
\date{2 November 2016}
\author{Advait Rajagopal}

\usepackage{amsmath}
\usepackage{latexsym}
\usepackage{graphicx}
\usepackage{amsfonts}
\usepackage{wasysym}
\usepackage{amssymb}
\usepackage{mathrsfs}
\usepackage{multirow,array}
\usepackage{booktabs}
\usepackage{float}

\usepackage[a4paper,bindingoffset=0.2in,%
      left=1in,right=1in,top=1in,bottom=1in,%
          footskip=.25in]{geometry}
\DeclareMathAlphabet{\mathpzc}{OT1}{pzc}{m}{it}
\linespread{1.3}
\usepackage{listings}
\usepackage[most]{tcolorbox}
\usepackage{inconsolata}
\newtcblisting[auto counter]{sexylisting}[2][]{sharp corners, 
    fonttitle=\bfseries, colframe=black, listing only, 
    listing options={basicstyle=\ttfamily,language=java}, 
    title=Listing \thetcbcounter: #2, #1}


\begin{document}
  \maketitle
\section{Question 1}
\textbf{Continuing the example from the previous assignment, discuss what can be done with the non-randomized part of the experiment (some classes get the Electric Company instead of the regular reading program and others get it in addition). Is there any evidence from the data that the replacement and supplemental assignments were assigned nonrandomly? }\\
In the previous assignment we ignored the fact that some treatments were assigned as \emph{replacements} and some as \emph{supplements} to the existing curriculum. My goal in this assignment is to first find out if the ``non-random" assignments of treatments (as replacement or supplement) is really non random, or how this property of the assignments impacts inference. If it is indeed non random I have to account for this in the model.\\
Figure 1 below shows the students who have been given the treatment in Fresno and Youngstown. The goal of this plot is to examine whether students of a certain type (certain pre test score) have received the supplemental treatment.
\begin{figure}[H]
\centering
\includegraphics[width = 16cm, height = 7cm]{repsup.png}
\caption{The pre test scores of the treated students in Fresno and Youngstown. The dots represent supplemental treatments and the circles represent replacements.}
\label{deltat}
\end{figure}

We see that there is no particular trend in the pre test scores that makes the teachers decide to give the students supplemental treatments instead of just replacements. We see that there is not really any evidence of non random replacement and supplement assignments. Now even though there is no evidence of a non random assignment of supplement and replacement conditional or pre test scores, city or grade, I move on to considering the actual impact of supplement instead of replacement as a treatment. It is important to remember that in this stage of the analysis we are not so interested in treatment vs. control performance but actually we want to know how post test scores vary as the\emph{type} of treatment varies as supplemental or replacement.\\ 
With this in mind I set up a hierarchical model with post test score as the dependent variable and pre test score and an indicator for supplement as predictors. These two predictors as in the previous assignment are allowed to vary by grade and by city. The full exposition of the model is given below.\\
\begin{align*}
y_i &\sim \textbf{N}(\alpha + \theta_{cg} S_i + \beta_{cg} x_i, \sigma^2)\\
%\end{align*}
%\begin{align*}
\beta_{cg} &\sim \textbf{N}(\mu_\beta, \tau_{\beta}^2)\\
\theta_{cg} &\sim
\begin{cases}
   \textbf{N}(\mu_{\theta F}, \tau_{\theta F}^2) ,& \text{if } c = \text{F}\\
     \textbf{N}(\mu_{\theta Y}, \tau_{\theta Y}^2) ,& \text{if } c = \text{Y}
\end{cases}
\end{align*}\\
Where $y_i$ is the post test score. $S_i$ and $x_i$ are indicators for supplement and the value of the pre test score respectively. $\alpha$ is the intercept, $\beta_{cg}$ is the coefficient that capture grade level effects of pre test score on the post test score and is given a normal prior distribution with mean $\mu_\beta$ and variance parameter $\tau_\beta^2$. The coefficient $\theta_{cg}$ is given a normal prior distribution that is city specific. Therefore supplemental treatments in city F (Fresno) come from a common distribution with a mean $\mu_{\theta F}$ and a variance $\tau_{\theta F}^2$. Similarly supplemental treatments in city Y (Youngstown) come from a distribution with mean $\mu_{\theta Y}$ and a variance $\tau_{\theta Y}^2$. I also use weakly informative prior distributions on the hyperparameters $\mu_{\theta F}$ , $\tau_{\theta F}$, $\mu_{\theta Y}$ and $\tau_{\theta Y}$. $\sigma$ is given a noninformative prior distribution.\\
Figure 2 shows the post scores of the students who received supplemental and replacement treatments and we see that the scores have improved as expected. The dots represent supplements and the circles represent replacements. The posterior estimates of the supplement parameter $\theta_{cg}$ as well as the hyperparameters of the distribution of pre test scores coefficient $\beta_{cg}$ are given in Table 1. The values converge with R-hat values close to 1. A close examination of Figure 2 reveals that students who received the supplement treatment have a much more substantial improvement in their scores than students who just received the replacement. This is perhaps because they were effectively taught the material twice. For example grade 2 students and grade 4 students in Fresno who got the supplement have drastic improvements in their score as compared to those who got replacements. Whereas as the posterior intervals in Table 1 show, the effect was not so strong in Youngstown and again the second graders help to make this point because some of the very poorly performing second graders have now caught up with the rest and those who were doing well have continued to do so. Even though we have no reason to believe that replacement and supplement are not assigned as randomly as possible. we must take into account the variation in the effectiveness of supplement versus replacement and particularly think about whether Fresno is really doing something different to ensure its supplemental treatments are very effective.

\begin{figure}[H]
\centering
\includegraphics[width = 14cm, height = 11cm]{Rplot.png}
\caption{A comparison of pre and post test scores across cities, across grades. Dots are supplemental treatments and circles are replacement treatments.}
\label{deltat}
\end{figure}
\begin{table} [H]
\caption {Posterior Means and SD of relevant parameters}
\vspace{2mm}
\def\arraystretch{1.5}
\centering \begin{tabular}{c c c c c c c c} 
\hline\hline 
\vspace{1mm}
 & mean&  sd   & 2.5\%  &  25\% &   50\% &   75\% & 97.5\%\\  [0.5ex] \hline
 $\theta_{1F}$ &  4.86 & 2.73  &  0.15  &  3.78  &  4.72  &  6.55  &  9.69\\
 $\theta_{1Y}$ &  1.66 & 2.71  &  -3.15  &  0.08  &  1.72  &  3.55  &  7.04\\
 $\theta_{2F}$ &  5.56 & 2.53  &  0.99  &  3.81  &  5.42  &  7.15  &  10.66\\
 $\theta_{2Y}$&  3.39 & 2.27  &  -1.15  &  1.98  &  3.32  &  4.85  &  8.12\\
$\theta_{3F}$&  4.20 & 2.66 &  -1.54 &    2.52 &   4.31 &   5.93  &  9.10\\
 $\theta_{3Y}$&1.36 & 2.74  & -4.45 &  -0.37  &  1.50 &   3.17  &  6.55       \\
  $\theta_{4F}$&4.34  &2.62 &  -0.91 &   2.80  &  4.33 &   6.03  &  9.38\\
 $\theta_{4Y}$&0.86 & 2.57 &  -4.26 &  -0.79  &  1.02 &   2.57 &   5.74\\
%$\mu_{\theta F}$    &    3.87 & 1.73  &  0.55  &  2.78  &  3.85  &  4.95  &  7.39   \\
%$\tau_{\theta F}$    &      2.26  & 1.44  &  0.22  &  1.16  &  2.07  &  3.11  &  5.55 \\
%$\mu_{\theta Y}$    &      4.10   & 2.25  & -1.08  &  2.72  &  4.19  &  5.58  &  8.39 \\
%$\tau_{\theta Y}$    &      3.84 &  2.15  &  0.58  &  2.25  &  3.57  &  4.98  &  9.03 \\
$\mu_{\beta}$     &     1.38 &  0.44   & 0.51  &  1.16  &  1.38  &  1.68   & 2.29  \\
$\tau_{\beta}$     &     1.14 & 0.41   & 0.65  &  0.92  &  1.10   & 1.37  &  2.18  \\
$\sigma$               &     6.03 &  0.32  &  5.45 &   5.82  &  6.01  &  6.23  &  6.69 \\
\hline 
\end{tabular}
\end{table}
It becomes clear from the table that there is a positive impact of including the supplemental treatment and moreover it appears the supplemental treatment is much stronger in Fresno than in Youngstown as the values of the coefficient $\theta_{cg}$ varying by grades and by city are consistently higher. Ideally we would like to introduce interaction terms between pre test scores and supplements to see what the impact of this on the post test score looks like, but this will make estimates unstable and noisy.
\begin{figure}[H]
\centering
\includegraphics[width = 14cm, height = 11cm]{final.png}
\caption{The relationship between pre and post test scores for treated classes is pictured here. Solid lines and dots show supplemental treatments. Dotted lines and circles show replacement treatments.}
\label{deltat}
\end{figure}
We observe that in general (with the exception of grade 1 at lower levels) the supplemental approach that involves using the TV show and the original curriculum is better because the solid line is always above the dotted line which represents replacements. We see a positive trend in both, but supplement clearly outperforms replacement in all grades.

\section{Code}
\subsection{R Code}
\begin{sexylisting}{R Code}
##Question 1
rm(list = ls())
setwd("/Users/Advait/Desktop/New School/Fall16/BDA/Class16")
library(rstan)
rstan_options(auto_write = TRUE)
options(mc.cores = parallel::detectCores())
library(plyr)
##
elec <- read.table("http://www.stat.columbia.edu/
~gelman/bda.course/electric.txt", header = T,skip  = 1)
str(elec)
# ##
# CREATING THE SNOOP
# ##
g1treat <- elec$Posttest[elec$Grade==1]
g1treat.pre <- elec$Pretest[elec$Grade==1]
g1cont <- elec$Posttest.1[elec$Grade==1]
g1cont.pre <- elec$Pretest.1[elec$Grade==1]
g1post <- c(g1treat,g1cont)
g1pre <- c(g1treat.pre,g1cont.pre)
t1 <- c(rep(1,21),rep(0,21))
city1 <- c(rep("F",11),rep("Y",10),rep("F",11),rep("Y",10))
g1 <- data.frame(g1pre,g1post,t1,city1)
names(g1) <- c("Pre", "Post", "Treat", "City")

#
g2treat <- elec$Posttest[elec$Grade==2]
g2treat.pre <- elec$Pretest[elec$Grade==2]
g2cont <- elec$Posttest.1[elec$Grade==2]
g2cont.pre <- elec$Pretest.1[elec$Grade==2]
length(g2treat)
g2post <- c(g2treat,g2cont)
g2pre <- c(g2treat.pre,g2cont.pre)
t2 <- c(rep(1,34),rep(0,34))
city2 <- c(rep("F",14),rep("Y",20),rep("F",14),rep("Y",20))
g2 <- data.frame(g2pre,g2post,t2,city2)
names(g2) <- c("Pre", "Post", "Treat", "City")
\end{sexylisting}
\begin{sexylisting}{R code contd.}
#
g3treat <- elec$Posttest[elec$Grade==3]
g3treat.pre <- elec$Pretest[elec$Grade==3]
g3cont <- elec$Posttest.1[elec$Grade==3]
g3cont.pre <- elec$Pretest.1[elec$Grade==3]
length(g3treat)
g3post <- c(g3treat,g3cont)
g3pre <- c(g3treat.pre,g3cont.pre)
t3 <- c(rep(1,20),rep(0,20))
city3 <- c(rep("F",10),rep("Y",10),rep("F",10),rep("Y",10))
g3 <- data.frame(g3pre,g3post,t3,city3)
names(g3) <- c("Pre", "Post", "Treat", "City")

#
g4treat <- elec$Posttest[elec$Grade==4]
g4treat.pre <- elec$Pretest[elec$Grade==4]
g4cont <- elec$Posttest.1[elec$Grade==4]
g4cont.pre <- elec$Pretest.1[elec$Grade==4]
length(g4treat)
g4post <- c(g4treat,g4cont)
g4pre <- c(g4treat.pre,g4cont.pre)
t4 <- c(rep(1,21),rep(0,21))
city4 <- c(rep("F",11),rep("Y",10),rep("F",11),rep("Y",10))
g4 <- data.frame(g4pre,g4post,t4,city4)
names(g4) <- c("Pre", "Post", "Treat", "City")

snoop <- rbind(g1,g2,g3,g4)
snoop <- data.frame(snoop)
str(snoop)
names(snoop) <- c("pre", "post", "treat","city")
snoop$grade <- c(rep(1,42),rep(2,68),rep(3,40), rep(4,42))
str(snoop)
snoop$supp <- elec$Supplement.
snoop$supp <- ifelse(snoop$supp=="S",1,0)
str(snoop)
plot(c(1:192),snoop$pre, col = "black")
snoop2 <- snoop
snoop2 <- snoop2[order(snoop2$city,
 snoop2$grade,snoop$supp),]
snoop2$sno <- c(1:192)
snoop2$city <- ifelse(snoop2$city=="F",1,2 )
\end{sexylisting}
\begin{sexylisting}{R code contd.}
##
par(mfrow = c(2,2),
    mar = c(3, 3, 1, 1), 
    oma = c(.5, .5, .5, .5), 
    mgp=c(2,1,0))
plot(snoop2$pre, 
     xlab = "Indexes",
     ylab = "Pre test score",
     main = "Treated students in Fresno", cex.main = 0.8,
     col = "white", xlim = c(1,95), ylim = c(0,120))
# legend(75,30, bty = "n",
#        legend = c("G1 Rep", "G2 Rep", "G3 Rep", "G4 Rep"),
#        col = c("black", "gray", "blue", "red"),pch = 1 , cex = 0.8)
# legend(60,30, bty = "n",
#        legend = c("G1 Sup", "G2 Sup", "G3 Sup", "G4 Sup"),
#        col = c("black", "gray", "blue", "red"), pch = 16, cex = 0.8)
points(snoop2$sno[snoop2$city==1 & 
snoop2$grade==1 & snoop2$supp==0 & snoop2$treat==1],
       snoop2$pre[snoop2$city==1 
       & snoop2$grade==1 & snoop2$supp==0 
       & snoop2$treat==1],cex = 0.8)
points(snoop2$sno[snoop2$city==1 & snoop2$grade==1 
& snoop2$supp==1 & snoop2$treat==1],
       snoop2$pre[snoop2$city==1 & snoop2$grade==1 
       & snoop2$supp==1 & snoop2$treat==1],pch = 16, cex = 0.8)

points(snoop2$sno[snoop2$city==1 & snoop2$grade==2 
& snoop2$supp==0 & snoop2$treat==1],
       snoop2$pre[snoop2$city==1 & snoop2$grade==2 
       & snoop2$supp==0 & snoop2$treat==1],cex = 0.8, 
       col = "gray")
points(snoop2$sno[snoop2$city==1 & snoop2$grade==2 & 
snoop2$supp==1 & snoop2$treat==1],
       snoop2$pre[snoop2$city==1 & snoop2$grade==2 & 
       snoop2$supp==1 & snoop2$treat==1],pch = 16, cex = 0.8,
       col = "gray")
\end{sexylisting}
\begin{sexylisting}{R code contd.}
points(snoop2$sno[snoop2$city==1 & snoop2$grade==3 
& snoop2$supp==0 & snoop2$treat==1],
       snoop2$pre[snoop2$city==1 & snoop2$grade==3
        & snoop2$supp==0 & snoop2$treat==1],cex = 0.8, col = "blue")
points(snoop2$sno[snoop2$city==1 & snoop2$grade==3 
& snoop2$supp==1 & snoop2$treat==1],
       snoop2$pre[snoop2$city==1 & snoop2$grade==3 
       & snoop2$supp==1 & snoop2$treat==1],pch = 16,
        cex = 0.8,
       col = "blue")

points(snoop2$sno[snoop2$city==1 & snoop2$grade==4 
& snoop2$supp==0 & snoop2$treat==1],
       snoop2$pre[snoop2$city==1 & snoop2$grade==4
        & snoop2$supp==0 & snoop2$treat==1],cex = 0.8, col = "red")
points(snoop2$sno[snoop2$city==1 & snoop2$grade==4 
& snoop2$supp==1 & snoop2$treat==1],
       snoop2$pre[snoop2$city==1 & snoop2$grade==4 
       & snoop2$supp==1 & snoop2$treat==1],pch = 16, cex = 0.8,
       col = "red")

plot(snoop2$pre, 
     xlab = "Indexes",
     ylab = "Pre test score",
     main = "Treated students in Youngstown", cex.main = 0.8,
     col = "white", xlim = c(93,192),ylim = c(0,120))
# legend(170,30, bty = "n",
#        legend = c("G1 Rep", "G2 Rep", "G3 Rep", "G4 Rep"),
#        col = c("black", "gray", "blue", "red"),pch = 1 , cex = 0.7)
# legend(155,30, bty = "n",
#        legend = c("G1 Sup", "G2 Sup", "G3 Sup", "G4 Sup"),
#        col = c("black", "gray", "blue", "red"), pch = 16, cex = 0.7)
points(snoop2$sno[snoop2$city==2 & snoop2$grade==1 
& snoop2$supp==0 & snoop2$treat==1],
       snoop2$pre[snoop2$city==2 & snoop2$grade==1
        & snoop2$supp==0 & snoop2$treat==1],cex = 0.8)
points(snoop2$sno[snoop2$city==2 & snoop2$grade==1 
& snoop2$supp==1 & snoop2$treat==1],
       snoop2$pre[snoop2$city==2 & snoop2$grade==1 
       & snoop2$supp==1 & snoop2$treat==1],
       pch = 16, cex = 0.8)
\end{sexylisting}
\begin{sexylisting}{R code contd.}
points(snoop2$sno[snoop2$city==2 & snoop2$grade==2 
& snoop2$supp==0 & snoop2$treat==1],
       snoop2$pre[snoop2$city==2 & snoop2$grade==2 
       & snoop2$supp==0 & snoop2$treat==1],cex = 0.8, col = "gray")
points(snoop2$sno[snoop2$city==2 & snoop2$grade==2 
& snoop2$supp==1 & snoop2$treat==1],
       snoop2$pre[snoop2$city==2 & snoop2$grade==2 
       & snoop2$supp==1 & snoop2$treat==1],pch = 16, cex = 0.8,
       col = "gray")

points(snoop2$sno[snoop2$city==2 & snoop2$grade==3 
& snoop2$supp==0 & snoop2$treat==1],
       snoop2$pre[snoop2$city==2 & snoop2$grade==3 
       & snoop2$supp==0 & snoop2$treat==1],cex = 0.8, col = "blue")
points(snoop2$sno[snoop2$city==2 & snoop2$grade==3 
& snoop2$supp==1 & snoop2$treat==1],
       snoop2$pre[snoop2$city==2 & snoop2$grade==3 
       & snoop2$supp==1 & snoop2$treat==1],pch = 16, cex = 0.8,
       col = "blue")

points(snoop2$sno[snoop2$city==2 & snoop2$grade==4
 & snoop2$supp==0 & snoop2$treat==1],
       snoop2$pre[snoop2$city==2 & snoop2$grade==4
        & snoop2$supp==0 & snoop2$treat==1],cex = 0.8, col = "red")
points(snoop2$sno[snoop2$city==2 & snoop2$grade==4 
& snoop2$supp==1 & snoop2$treat==1],
       snoop2$pre[snoop2$city==2 & snoop2$grade==4 & 
       snoop2$supp==1 & snoop2$treat==1],pch = 16, cex = 0.8,
       col = "red")

#################################

# par(mfrow = c(1,2),
#     mar = c(3, 3, 1, 1), 
#     oma = c(.5, .5, .5, .5), 
#     mgp=c(2,1,0))
plot(snoop2$post, 
     xlab = "Indexes",
     ylab = "Post test score",
     main = "Post scores in Fresno", cex.main = 0.8,
     col = "white", xlim = c(1,95),ylim = c(0,120))
     \end{sexylisting}
\begin{sexylisting}{R code contd.}
# legend(75,30, bty = "n",
#        legend = c("G1 Rep", "G2 Rep", "G3 Rep", "G4 Rep"),
#        col = c("black", "gray", "blue", "red"),pch = 1 , cex = 0.8)
# legend(60,30, bty = "n",
#        legend = c("G1 Sup", "G2 Sup", "G3 Sup", "G4 Sup"),
#        col = c("black", "gray", "blue", "red"), pch = 16, cex = 0.8)
points(snoop2$sno[snoop2$city==1 & snoop2$grade==1 
& snoop2$supp==0 & snoop2$treat==1],
       snoop2$post[snoop2$city==1 & snoop2$grade==1 
       & snoop2$supp==0 & snoop2$treat==1],cex = 0.8)
points(snoop2$sno[snoop2$city==1 & snoop2$grade==1 
& snoop2$supp==1 & snoop2$treat==1],
       snoop2$post[snoop2$city==1 & snoop2$grade==1 
       & snoop2$supp==1 & snoop2$treat==1],pch = 16, cex = 0.8)

points(snoop2$sno[snoop2$city==1 & snoop2$grade==2 
& snoop2$supp==0 & snoop2$treat==1],
       snoop2$post[snoop2$city==1 & snoop2$grade==2 
       & snoop2$supp==0 & snoop2$treat==1],cex = 0.8, col = "gray")
points(snoop2$sno[snoop2$city==1 & snoop2$grade==2 
& snoop2$supp==1 & snoop2$treat==1],
       snoop2$post[snoop2$city==1 & snoop2$grade==2 
       & snoop2$supp==1 & snoop2$treat==1],pch = 16, cex = 0.8,
       col = "gray")

points(snoop2$sno[snoop2$city==1 & snoop2$grade==3 
& snoop2$supp==0 & snoop2$treat==1],
       snoop2$post[snoop2$city==1 & snoop2$grade==3 
       & snoop2$supp==0 & snoop2$treat==1],cex = 0.8, col = "blue")
 points(snoop2$sno[snoop2$city==1 & snoop2$grade==3 
 & snoop2$supp==1 & snoop2$treat==1],
       snoop2$post[snoop2$city==1 & snoop2$grade==3 
       & snoop2$supp==1 & snoop2$treat==1],pch = 16, cex = 0.8,
       col = "blue")

points(snoop2$sno[snoop2$city==1 & snoop2$grade==4 
& snoop2$supp==0 & snoop2$treat==1],
       snoop2$post[snoop2$city==1 & snoop2$grade==4 
       & snoop2$supp==0 & snoop2$treat==1],cex = 0.8, col = "red")
       \end{sexylisting}
\begin{sexylisting}{R code contd.}
points(snoop2$sno[snoop2$city==1 & snoop2$grade==4 
& snoop2$supp==1 & snoop2$treat==1],
       snoop2$post[snoop2$city==1 & snoop2$grade==4
        & snoop2$supp==1 & snoop2$treat==1],pch = 16, cex = 0.8,
       col = "red")

plot(snoop2$post, 
     xlab = "Indexes",
     ylab = "Post test score",
     main = "Post scores in Youngstown", cex.main = 0.8,
     col = "white", xlim = c(93,192),ylim = c(0,120))
# legend(170,30, bty = "n",
#        legend = c("G1 Rep", "G2 Rep", "G3 Rep", "G4 Rep"),
#        col = c("black", "gray", "blue", "red"),pch = 1 , cex = 0.7)
# legend(155,30, bty = "n",
#        legend = c("G1 Sup", "G2 Sup", "G3 Sup", "G4 Sup"),
#        col = c("black", "gray", "blue", "red"), pch = 16, cex = 0.7)
points(snoop2$sno[snoop2$city==2 & 
snoop2$grade==1 & snoop2$supp==0 & snoop2$treat==1],
       snoop2$post[snoop2$city==2 & 
       snoop2$grade==1 & snoop2$supp==0 
       & snoop2$treat==1],cex = 0.8)
points(snoop2$sno[snoop2$city==2 & s
noop2$grade==1 & snoop2$supp==1 & snoop2$treat==1],
       snoop2$post[snoop2$city==2 & 
       snoop2$grade==1 & snoop2$supp==1 
       & snoop2$treat==1],pch = 16, cex = 0.8)

points(snoop2$sno[snoop2$city==2 & 
snoop2$grade==2 & snoop2$supp==0 
& snoop2$treat==1],
       snoop2$post[snoop2$city==2 & 
       snoop2$grade==2 & snoop2$supp==0 & s
       noop2$treat==1],cex = 0.8, col = "gray")
points(snoop2$sno[snoop2$city==2 & 
snoop2$grade==2 & snoop2$supp==1 & snoop2$treat==1],
       snoop2$post[snoop2$city==2 & snoop2$grade==2 
       & snoop2$supp==1 & snoop2$treat==1],
       pch = 16, cex = 0.8,
       col = "gray")
\end{sexylisting}
\begin{sexylisting}{R code contd.}
points(snoop2$sno[snoop2$city==2 & snoop2$grade==3 & 
snoop2$supp==0 & snoop2$treat==1],
       snoop2$post[snoop2$city==2 & snoop2$grade==3 & 
       snoop2$supp==0 & snoop2$treat==1],cex = 0.8, col = "blue")
points(snoop2$sno[snoop2$city==2 & snoop2$grade==3 & 
snoop2$supp==1 & snoop2$treat==1],
       snoop2$post[snoop2$city==2 & snoop2$grade==3 & 
       snoop2$supp==1 & snoop2$treat==1],pch = 16, cex = 0.8,
       col = "blue")

points(snoop2$sno[snoop2$city==2 & snoop2$grade==4 
& snoop2$supp==0 & snoop2$treat==1],
       snoop2$post[snoop2$city==2 & snoop2$grade==4 &
        snoop2$supp==0 & snoop2$treat==1],
        cex = 0.8, col = "red")
points(snoop2$sno[snoop2$city==2 & snoop2$grade==4 
& snoop2$supp==1 & snoop2$treat==1],
       snoop2$post[snoop2$city==2 & snoop2$grade==4 
       & snoop2$supp==1 & snoop2$treat==1],pch = 16, cex = 0.8,
       col = "red")

####################
#Stan Model
####################
snoop3 <- subset(snoop2, snoop2$treat==1)
pre <- snoop3$pre
post <- snoop3$post
city <- snoop3$city
grade <- snoop3$grade
supp <- snoop3$supp
length(supp)

stanc("9a.stan")
fit_sup <- stan("9a.stan", 
                data = list("pre", "post", "supp", 
                            "city", "grade"),
                iter = 1000,
                chains = 3)
print(fit_sup)
##########################
\end{sexylisting}
\begin{sexylisting}{R code contd.}
par(mfrow = c(2, 2), mar=c(3, 3, 2, 1), 
    oma=c(.5, .5, .5, .5), mgp=c(2, 1, 0))
for(i in 1:4){
  plot(snoop$pre[snoop$grade==i & snoop$treat==1 &
   snoop$supp=="S"],snoop$post[snoop$grade==i & 
   snoop$treat==1 & snoop$supp=="S"], 
  pch = 16, cex = 0.8, xlab = "Pre test score", 
  ylab = "Post test score", main = paste("Grade",i), 
  xlim = c(0,120), ylim = c(0,120))
  points(snoop$pre[snoop$grade==i & snoop$treat ==1 & 
  snoop$supp=="R"],snoop$post[snoop$grade==i & 
  snoop$treat ==1 & snoop$supp=="R"] ,cex = 0.8)

abline(lm(snoop$post[snoop$grade==i & snoop$treat==1 & 
                       snoop$supp=="S"] ~ 
                       snoop$pre[snoop$grade==i & 
                       snoop$treat==1 & snoop$supp=="S"]))

abline(lm(snoop$post[snoop$grade==i 
& snoop$treat==1 & 
                       snoop$supp=="R"] ~ 
                       snoop$pre[snoop$grade==i & 
                       snoop$treat==1 & snoop$supp=="R"]), lty = 2)
        
}
\end{sexylisting}
\subsection{Stan Code}
\begin{sexylisting}{Stan Code}

data{
  vector[96]post;
  vector[96]pre;
  vector[96]supp;
  int grade[96];
  int city[96];
}
parameters{
  real b0;
  real beta [4,2];
  matrix [4,2] theta ;
  real <lower = 0> sigma;
  real mu_F;
  real<lower = 0> tau_F;
  real mu_Y;
  real<lower = 0> tau_Y;
  real mu_beta;
  real<lower = 0> tau_beta;
}
model{
  for(i in 1:96){
    post[i] ~ normal(b0 + beta[grade[i], city[i]]*pre[i] + 
                          theta[grade[i], city[i]]*supp[i] ,
                          sigma);
  }
  theta[, 1] ~ normal(mu_F, tau_F);
  theta[, 2] ~ normal(mu_Y, tau_Y);
  mu_F ~ normal(5,5);
  mu_Y ~ normal(5,5);
  tau_F ~ normal(3,2);
  tau_Y ~ normal(3,2);
  for(i in 1:4){
    beta[i,] ~ normal(mu_beta, tau_beta);
  }
}









\end{sexylisting}

\end{document}